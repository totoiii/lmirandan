% Preamble
% Compile with XeLateX

\documentclass[11 pt,oneside,a4paper,titlepage]{article}
\usepackage{preamble}
\graphicspath{{PIC/}}
%%%%%%%%%%%%%%%%%%%%%%%%%%%%%%%%%%%%%%%%%%%%%%%%%%%%%%%%%%%%%%%%%%%%%%%%%%%%%%%%%%%%%%
\begin{document}

\sidebar{sideBarColor!25}
\simpleheader{titleBackColor}{Luis}{Miranda Navarro}{PhD Student at \href{https://www.uc.cl/}{PUC}\hspace{1mm} \faLightbulbO \hspace{1mm} PhD in Computer Science, area Clinical NLP and Data Privacy}{white}

% Start Minipages
\vspace*{3.49cm}% start 8 cm from the top of the page}
    \adjustbox{valign=t}{\begin{minipage}{7.3cm} % large 7.4 cm from the top
    \vspace*{1.2cm} % text starts 1cm under the top of the minipage
            
         %Picture
        %\begin{center}
        %\begin{tikzpicture}
            %\node[
            %circle,
            %minimum size=\cvPictureWidth,
            %path picture={
            %\node at (path picture bounding box.center){
             %\includegraphics[width=\cvPictureWidth]{PIC/perfil.jpeg}
             %};
             %}]
            %{};
        %\end{tikzpicture}
        %\end{center}

        %%%%%%%%%%%%%%%%%%%%%%%%%%%%%%%%%%%%%%%%%%%%%%%%%%%%
        % Profile section
        \ruleline{\textbf{About me}}
       PhD student in Computer Science, specializing in Natural Language Processing and Data Privacy in the Clinical Domain, and holding a Bachelor's degree in Physics from UC. Currently employed as a researcher and data scientist at various institutions, including the Chilean Safety Association (\href{https://achs.cl/}{ACHS}) and the Millenium Institute for Foundational Research on Data (\href{https://imfd.cl/}{IMFD}). Also, a member of the Natural Language Processing research group at the Center for Mathematical Modeling (\href{https://github.com/plncmm/}{PLNCMM}). Has worked as a teaching assistant since the beginning of their career in physics subjects, acquiring soft skills and teaching abilities. Proficient in Python, artificial intelligence, and software engineering. Developed self-taught and research skills in NLP and programming, having co-authored 2 papers currently under review. Enjoys sports such as swimming and trekking.
        
        %%%%%%%%%%%%%%%%%%%%%%%%%%%%%%%%%%%%%%%%%%%%%%%%%%%
        % Contact Section
        \ruleline{\textbf{Personal Info}}
        \begin{tikzpicture}[every node/.style={inner sep=0pt, outer sep=0pt}]
        \matrix [
        column 1/.style={anchor=center,contactIcon},
        column 2/.style={anchor=west,align=left,contactIcon},
        column sep=5pt,
        row sep=5pt] (contact) {
        \node{\faMale};
         & \node{May 14th, 2001.  22 years old.};\\
        \node{\faEnvelope}; 
         & \node{\href{mailto:lmirandn@uc.cl}{lmirandn@uc.cl}};\\
         \node{\faEnvelopeO}; 
        \node{\faPhone}; 
         & \node{+56 9 3003 3502};\\ 
        %\node{\faLinkedinSquare}; 
        \\
        %\node{\aiResearchGateSquare}; 
        %& \node{\href{your Research Gate Link here}{Research Gate: Jack}};\\
        %\node{\aiOrcid}; 
        %& \node{\href{your Orcid Link here}{ORCID: xxxx-xxxx-xxxx-xxxx}};\\
        %\node{\faCar};
        %& \node{Licencia de Conducir Clase B};\\
         };
        \end{tikzpicture} 
        \ruleline{\textbf{Skills}}
    \vspace*{-0.5cm}
    \begin{center}
        \cvtag{Creativity}\cvtag{Curiosity}\cvtag{Team-work}\cvtag{Teaching}\cvtag{Autonomy} \cvtag{Perseverance}
    \end{center}
        %%%%%%%%%%%%%%%%%%%%%%%%%%%%%%%%%%%%%%%%%%%%%%%%%%%
        \ruleline{\textbf{Language}}
        \begin{tikzpicture}[every node/.style={inner sep=0pt, outer sep=0pt}]
        \matrix [
        column 1/.style={anchor=center,contactIcon},
        column 2/.style={anchor=west,align=left,contactIcon},
        column sep=5pt,
        row sep=5pt] (contact) {
        \node{\flag{England.png}};
        & \node{English - Intermediate};\\
        %\node{\flag{Italy.png}};
        %& \node{Italian - Professional Knowledge};\\
        \node{\flag{Spain.png}};
        & \node{Spanish - Native};\\
        %\node{\flag{France.png}};
        %& \node{French - Professional Knowledge};\\
        };
        \end{tikzpicture} 
        \ruleline{{\faCode} \textbf{Programming Languages}}
        \begin{tikzpicture}[every node/.style={inner sep=0pt, outer sep=0pt}]
        \matrix [
        column 1/.style={anchor=center,contactIcon},
        column 2/.style={anchor=west,align=left,contactIcon},
        column sep=5pt,
        row sep=5pt] (contact) {
        \\node{};
        & \node{\textbf{Matlab}: Basic };\\
        %\node{\flag{Italy.png}};
        %& \node{Italian - Professional Knowledge};\\
        \node{};
        & \node{\textbf{Python}: Advanced (Keras, Pytorch, \\
        
        Tensorflow, Hugging-Face, Langchain, etc)};\\
        
        %\node{\flag{France.png}};
        %& \node{French - Professional Knowledge};\\
        \node{};
        & \node{\textbf{C}: Advanced};\\
        };
        \end{tikzpicture} 
        \ruleline{{\faDesktop} \textbf{Technologies}}
        \begin{tikzpicture}[every node/.style={inner sep=0pt, outer sep=0pt}]
        \matrix [
        column 1/.style={anchor=center,contactIcon},
        column 2/.style={anchor=west,align=left,contactIcon},
        column sep=5pt,
        row sep=5pt] (contact) {
        \node{Data Analysis};
        & \node{\textbf{Matlab}: Basic };\\
        %\node{\flag{Italy.png}};
        %& \node{Italian - Professional Knowledge};\\
        & \node{\textbf{Wolfram Math}: Advanced};\\
        & \node{\textbf{Jupyter Notebook}: Advanced};\\
        %\node{\flag{France.png}};
        %& \node{French - Professional Knowledge};\\
        \node{Office};
        & \node{\textbf{\LaTeX}: Advanced} ;\\
        & \node{\textbf{MS Office}: Advanced} ;\\
        };
        \end{tikzpicture} 
        %%%%%%%%%%%%%%%%%%%%%%%%%%%%%%%%%%%%%%%%%%%%%%%%%%%%%
        % QR Code
        \ruleline{\textbf{Courses}}
        \begin{tikzpicture}[every node/.style={inner sep=0pt, outer sep=0pt}]
        \matrix [
        column 1/.style={anchor=center,contactIcon},
        column 2/.style={anchor=west,align=left,contactIcon},
        column sep=5pt,
        row sep=5pt] (contact) {
        \node{};
        & \node{Software engineering, Artificial Intelligence,\\ Computer Vision, Recommender Systems, \\Deep Learning, Natural language Processing, };\\
        %\node{\flag{Italy.png}};
        %& \node{Italian - Professional Knowledge};\\
        \node{};
        & \node{Data Structure and algorithms, Data Privacy};\\
        %\node{\flag{France.png}};
        %& \node{French - Professional Knowledge};\\
        };
        \end{tikzpicture}
        
        
    \end{minipage}} %
    \hfill 
%%%%%%%%%%%%%%%%%%%%%%%%%%%%%%%%%%%%%%%%%%%%%%%%%%%%%%%%%
%%%%% MAIN SECTION %%%%%%%%%%%%%%%%%%%%
    \adjustbox{valign=t}{\begin{minipage}{11.3cm}
        \vspace*{1cm}
        \section*{{\faGraduationCap}Education}

        \MySection{August 2023 - Today}{}{PhD in Engineering Sciences, area Computer Science}{Pontificia Universidad Católica de Chile}{Vicuña Mackenna 4860, Macul}{\href{https://dcc.ing.puc.cl/}{Department of Computer Science, PUC}}
            {Currently in my second semester of doctoral studies, under the supervision of Professor \href{https://sites.google.com/view/jdunstan/home}{Jocelyn Dunstan}}.
        \vspace*{0.12cm}
            
        \MySection{2019-2023}{}{Bachelor's Degree in Physics}{Pontificia Universidad Católica de Chile}{Vicuña Mackenna 4860, Macul}{Faculty of Physics, PUC}{I hold a Bachelor's Degree in Physics from PUC with excellent grades, even receiving the award for the highest GPA in 2021.}{}
        \vspace*{0.12cm}        
        
                
        %%%%%%%%%%%%%%%%%%%%%%%%%%%%%%%%%%%%%%%%%%%%%%%%%%%
        % Work Experience
        \section*{{\faSuitcase} Work Experience}
        \vspace*{0.22cm} 
         \MySectionNoPic{August 2023 - Today}{Clinical NLP Data Scientist}{Ramón Carnicer 163}{Chilean Safety Association (\href{https://achs.cl/}{ACHS})}{I work as a researcher at ACHS, specializing in the field of clinical NLP, creating language models and corpora to address classification and NER tasks.}

         \MySectionNoPic{August 2023 - Today}{Data privacy Researcher}{Vicuña Mackenna 4860, Macul}{Millenium Institute for Foundational Research on Data (\href{https://imfd.cl/}{IMFD})}{I am part of the Data Privacy group at IMFD, currently researching the creation and evaluation of language models that ensure privacy guarantees.}
       
         
        \MySectionNoPic{2020-2023}{Undergraduate Research}{Vicuña Mackenna 4860}{Facultad de Física UC}{During my physics studies, I worked on computational simulations of atomic systems (NAVY project), simulations of quantum systems, and quantum metrology using ESR.}
            
        
        
        \MySectionNoPic{2019-2022}{Teaching Assistantships}{Vicuña Mackenna 4860}{Facultad de Física UC}{Theoretical and laboratory teaching assistantships in courses such as "Dynamics," "Thermodynamics," and "Classical Mechanics" at the Faculty of Physics, PUC.} 

        
         
        \vspace*{0.22cm}    
       

     \section*{Publications and Participation in Research}
         \vspace*{0.5cm}  
    I have been actively involved in several research endeavors. Notably, I contributed to the development of a Python library, and I co-authored both papers that are currently undergoing review.
    \begin{itemize}

    \footnotesize
        
        \item Research Article: \textbf{\textit{A pseudonymized corpus of occupational health narratives for clinical entity recognition in Spanish}} (under review). The preprint is available \href{https://doi.org/10.21203/rs.3.rs-3826527/v1}{here}.
        \item Research Article: \textbf{\textit{A Privacy-Preserving Corpus for Occupational Health in Spanish: Evaluation for NER and Classification Tasks}} (under review). This paper presents a corpus annotated to anonymize personally identifiable information. The value of the corpus is demonstrated by evaluating its use for NER and a classification task. Submitted to the NAACL 2024 Workshop on Clinical NLP.
        \item Python Library: \textbf{llmNER}, library for implementing zero-shot and few-shot Named Entity Recognition with LLMs, the library enables the user to perform prompt engineering efficiently by multiple prompting methods, answer shapes, few-shot demonstrations, and prompt templates. See \href{https://github.com/plncmm/llmner}{\texit{llmNER's github}}.
        \item[\vspace{\fill}]
    \end{itemize}
%%%%%%%%%%%%%%%%%%%%%%%%%%%%%%%%%%%%%%%%%%%%%%%%%%%
        % Publications
        %\section*{{\aiOBP} PUBLICACIONES}
            
        %\publication{Journal Article}{2022}{Estudio de los dobles caminos producidos por
%interferencias en el telescopio de banda Q de CLASS a 38GHz}{Rodrigo Pozo}{Journal I3, Ingeniería UC}{}
            
    \end{minipage}} %

%%%%%%%%%%%%%%%%%%%%%%%%%%%%%%%%%%%%%%%%%%%%%%%%%%%%%%%%%%%%
% Second Page



\end{document}
